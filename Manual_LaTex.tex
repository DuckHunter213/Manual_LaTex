%De que tipo de documento se habla.
\documentclass[10pt,a4paper]{article}

%Importe de los paquetes a ocupar.
\usepackage{amsmath}
\usepackage[spanish]{babel}
%\usepackage[utf8]{inputenc}
\usepackage{ragged2e}
%Aquí se definen las dimensiones de la hoja.
\setlength{\textwidth}{15cm} \setlength{\textheight}{25cm}
\setlength{\hoffset}{-.5cm} \setlength{\oddsidemargin}{1cm}
\setlength{\evensidemargin}{1cm} \setlength{\topmargin}{-2cm}
\pagestyle{empty}

% Escribiendo lás referencias de los autores.
\addto\captionsspanish{\def\bibname{Referencias}}
\renewenvironment{thebibliography}[1]
     {\footnotesize \section*{\normalsize \bibname}
      \list{[\arabic{enumi}]}{\settowidth\labelwidth{[#1]}
            \leftmargin\labelwidth
            \advance\leftmargin\labelsep
            \usecounter{enumi}}
            }

\usepackage{configuracion}
\usepackage{portada}
%\subinput*{../}{configuracion.tex}

\begin{document}

\Portada


\titulo{Introducción}

\titulo{Inicio Básico}
\titulo{Comandos}

\comando{usepackage}{Este comando es el encargado de importa librerias al documento, es decir una agrupación de otros comandos diseñado para reducir lineas de código y empaquetar funciones}
\ejemplo{usepackage[spanish]{babel}:  Para importar la escritura con tildes}
\ejemplo{usepackage[utf8]{inputenc}: }

\subtitulo{Formatos de texto}

\comando{textbf}{Este comando es ocupado para poner letras en negritas}{\textbf{Negritas}}

\comando{textit}{Este comando es ocupado para poner letras en cursivas}{\textit{Cursivas}}


\titulo{Contenido Avanzado}


\end{document}